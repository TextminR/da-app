\section{Visualisierung der Textdaten}\label{sec:visualisierung}

"Nach der Definition der Problemstellungen und Ziel soll recherchiert werden, wie diese erreicht, beziehungsweise gelöst werden können. Diese Studie beschäftigt sich mit möglichen Lösungen und Technologien und analysiert deren Eigenschaften um konkrete Vor- und Nachteile zu finden. Beendet wird dieser Abschnitt mit einem Fazit."

\footnote{\url{"url"}} \nameref{sec:*}

\smallskip\par\noindent

In diesem Abschnitt wird die Visualisierung der Textdaten behandelt. Die Herausforderung besteht darin, eine benutzerfreundliche Web-Applikation zu entwickeln, die die bereits aufbereiteten Daten auf möglichst übersichtliche Weise präsentiert. Hierbei stehen verschiedene Technologien zur Auswahl, die im folgenden Abschnitt eingehend erörtert und analysiert werden. Am Ende dieses Prozesses wird die am besten geeignete Methode identifiziert und ihre Auswahl begründet.

\subsection{Ziele}

\subsubsection{...}

...

\subsection{Technologien}

\subsubsection{Shiny}

Shiny stellt eine Open-Source-Softwarebibliothek dar, welche speziell für die Programmiersprache R entwickelt wurde. Die Software wurde von RStudio konzipiert und hat den Zweck, interaktive, auf R-Code basierende Webanwendungen zu generieren.

\subsubsection{Dash (Plotly Dash)}

\subsubsection{Bokeh}

\subsubsection{Flexdashboard}

\subsubsection{Tableau}

\subsubsection{Django und Flask}

\subsubsection{JavaScript-basierte Frameworks}

